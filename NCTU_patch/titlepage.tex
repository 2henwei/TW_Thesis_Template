\documentclass[class=NCU_thesis, crop=false, float=true]{standalone}
\begin{document} 
% I use LaTeX3 to automatically generate name table. 
% Below \ExplSyntaxOn to \ExplSyntaxOff perpare prof. table contents,
% it will save contents to `\profsTableContent''. 
% You can ignore this block even you want make table by yourself.
\ExplSyntaxOn
% Copy prof. list from config.tex
\clist_gclear_new:N \g_sppmg_profsZh_cl
\clist_gset:NV \g_sppmg_profsZh_cl \profsZh
\clist_gclear_new:N \g_sppmg_profsEn_cl
\clist_gset:NV \g_sppmg_profsEn_cl \profsEn

% get total number of prof. . Omitted language will not display.
\int_gzero_new:N \g_sppmg_profTotal_int
\int_gset:Nn \g_sppmg_profTotal_int {
    \int_max:nn {\clist_count:N \g_sppmg_profsZh_cl} 
        {\clist_count:N \g_sppmg_profsEn_cl}
}

% NOTE: ``tabularx'' will  processes its contents more than once 
% for calculate width, so ``gpop'' can't put in tabularx env.
% \tl_if_exist:NTF {\tl_clear:N \g_sppmg_tableContent_tl} {\tl_new:N \g_sppmg_tableContent_tl}
% \tl_if_exist:NTF \g_sppmg_tableContent_tl {} {\tl_new:N \g_sppmg_tableContent_tl}
\tl_gclear_new:N \g_sppmg_tableContent_tl


% Use a inline function for pop 2 list (zh+en), and save table content 
% Input(#1) switch 3 case, 1 = Advisor, 2 = committee member , 3+ is more.
% Use ``for'' loop to get all prof.
\int_step_inline:nnnn {1}{1}{\g_sppmg_profTotal_int}{
    \clist_gpop:NNTF \g_sppmg_profsZh_cl \l_tmpa_tl {}{ \tl_clear:N \l_tmpa_tl}
    \clist_gpop:NNTF \g_sppmg_profsEn_cl \l_tmpb_tl {}{ \tl_clear:N \l_tmpb_tl}
    \tl_gput_right:Nx \g_sppmg_tableContent_tl {
        \int_case:nnTF {#1}{
            {1} {指導教授: & \l_tmpa_tl & &  Advisor:  & \l_tmpb_tl \exp_not:n {\\} }
            {2} {共同指導: & \l_tmpa_tl & &  Co-advisor:  & \l_tmpb_tl \exp_not:n {\\} }
        }{}{
            & \l_tmpa_tl & &   & \l_tmpb_tl  \exp_not:n {\\} 
        }
    }
}

% Copy contents to LaTeX2e macro.
\cs_set_eq:NN \profsTableContent \g_sppmg_tableContent_tl

\ExplSyntaxOff
    
    
% \def\fsUniversityZh{\fontsize{18}{27}\selectfont }
% \def\fsUniversityEn{\fontsize{16}{24}\selectfont }
% \def\fsDeptEn{\fontsize{14}{21}\selectfont }
% \def\fsTitle{\fontsize{18}{27}\selectfont }
% \def\fsNames{\fontsize{18}{27}\selectfont }
% --------define title page layout for thesis
\titlepageFontFamily % set in config.tex
\newgeometry{top=2.0cm, bottom=2.0cm, inner=2cm, outer=2cm} % only for titlepage
\begin{spacing}{1.0}
\begin{titlepage}
%     \null\vfill
    \begin{center}
        {\fs{16} {\titleZh} \par}
        \vspace*{5mm}
        {\fs{16} {\titleEn} \par}
        \vspace*{30mm}
        
        {\fs{14}[1.5] \renewcommand{\arraystretch}{1}
            % If you want make table by yourself, replace ``\profsTableContent''
            \begin{tabularx}{\textwidth}{l@{\hspace*{0.4em}}lXr@{\hspace*{0.4em}}l}
                % 0.4em for ``:'' right side space
                研\enspace 究\enspace 生: & \authorZh  &  & Student: & \authorEn   \\
                
                \profsTableContent
                
            \end{tabularx}
        \par}
        \vspace*{15mm} \vfill
        
        {\fs{14}[20]
        國立交通大學 \par
        \deptZh \par
        \degreeZh 論文\par}
        \vspace*{20mm} \vfill
        
        {\fs{14}[18] 
        A Thesis \par
        Submitted to \deptEn \par
        \ifx \collegeEn\empty\else College of \collegeEn  \par \fi 
        National Chiao Tung University \par
        in partial Fulfillment of the Requirements \par
        for the Degree of \par
        \degreeEn \par
        in \par
        \vspace*{10mm} \vfill
        
        \deptEn 
        \vspace*{10mm} \vfill
        
        \degreedateEn \par
        \vspace*{10mm} \vfill
        
        Hsinchu, Taiwan, Republic of China \par
        \vspace*{10mm} \vfill
        \degreedateROC}
    \end{center}
\end{titlepage}
\end{spacing}
\restoregeometry
\normalfont % use main font
%--------end of title page for thesis
\cleardoublepage

\end{document}
