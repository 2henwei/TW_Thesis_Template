%%%%% build setting | 編譯設定 %%%%%
\setboolean{publish}{false} % {true}/{false} Set true before publish. 發佈前設true
\def\lang{zh} % {zh}/{en} Set main language.
\setboolean{disableChinese}{false} % {true}/{false} Disable Chinese, for English user who don't have Chinese fonts.

\synctex=1 % Enable SyncTeX


%%%%% Information of your document. | 定義文件資訊 %%%%%
\def\deptshort {物理}
\def\deptZh    {物理研究所} % (請參考各校規定)
\def\deptEn    {Department of Physics}
\def\collegeEn {}
\def\degreeZh  {碩士} % 碩士/博士
\def\degreeEn  {Master} % Master/博士 | master thesis/doctoral dissertation
\def\titleZh   {交通大學LaTeX論文樣板(中文)}   % Chinese title
\def\titleEn   {NCTU LaTeX Thesis Template(Chinese)} % English title
\def\title     {\titleZh} % Main title, default Chinese title
\def\subtitle  {\titleEn} % Subtitle, default English title, empty allowed.
\def\logo      {}   % logo on cover. Watermark see below.
\def\authorZh  {君の名は。}
\def\authorEn  {Your name.}
\def\author    {\authorZh} % Choose one When only use one name.
\def\mprofZh   {你的指導教授}
\def\sprofiZh  {你的共同指導} % 共同指導 1
\def\sprofiiZh {}      % 共同指導 2                  
\def\mprofEn   {Your committee}
\def\sprofiEn  {Committee member} % 共同指導 1
\def\sprofiiEn {}      % 共同指導 2   
\def\degreedateROC{中~華~民~國~一百零七~年~六~月}
\def\degreedateEn {Jun, 2018}
\def\degreeyearROC{106} % academic year of ROC
\def\degreesemesterROC{2}
\def\copyyear  {2012-2013}
% \def\doi       {10.1006/jmbi.1998.2354} % Do not add ``doi:``

\setboolean{printcopyright}{false} % {true}/{false} print copyright text on titlepage or cover.

\def\keywordsZh{關鍵字, 論文, 樣板, 讓我畢業} % don't use \def\keywords
\def\keywordsEn{Keyword, Thesis, Template, Graduate me}


%%%%% Set letters filename | 設定插入文件 %%%%%
% This setting for main.tex default letters.
% It will not insert PDF if file non-exist or empty .
% 此為 main.tex 預設插入之 PDF,檔案不存在或空白則不插入。

% 交大:授權書和審定書不須加入PDF檔,此處樣板不處理。

% 碩博士論文電子檔授權書 Authorization Letter (for electronic)
% \def\letterAuthEl{letter_authorization.pdf}
% 碩博士紙本論文延後公開/下架申請書。(如需延後公開者,才需要裝訂於論文內頁)
% \def\letterPubReq{letter_publication_request.pdf}
% 指導教授推薦書
% \def\letterRecom {letter_recommendation.pdf}
% 口試委員審定書
% \def\letterVerif {letter_verification.pdf}


%%%%% Bibliography | 文獻列表 %%%%%
\def\bibManType{2} % {0}/{1}/{2} 0 = Embedded, 1 = BibTeX, 2 = biber / BibLaTeX
\def\bibStyle{ieee} % Default ``ieee'' with BibLaTeX. If you use BibTeX change to ``ieeetr''.
% BibLaTeX ref: https://www.sharelatex.com/learn/Biblatex_bibliography_styles
% BibTeX ref: https://www.sharelatex.com/learn/Bibtex_bibliography_styles
\setboolean{bibStyleNameYear}{false} % {true}/{false} true for use name,year to sort and cite.(If you want use custom option in .cls, set false here. )


%%%%% Set OS | 設定作業系統 %%%%%
% It will overwrite \OS if your LaTeX compile parameter has ``-shell-escape''(Texstudio default enable).
% Linux user: keep setting or add ``-shell-escape'' to compiler.
% Mac OS X user: Set to mac or add ``-shell-escape'' to compiler.
% Windows user: Don't need change setting.
\def\OS{linux} % {linux}/{mac}/{win}, only effect auto select CJK font.(CJK means Chinese, Japanese, and Korean)


%%%%% Set font | 設定中英文字型 %%%%%
% Keep empty for default font. CJK font must set OS for auto select.
% Setting by name of font(English will be better) or font filename.
% Linux 利用指令 fc-list :lang=zh 來查詢可以用的字體名稱。
% Windows 使用內建字型管理查詢。建議填英文字型名稱。
\def\mainfont   {} % default use Latin Modern Roman (lmodern pkg.)
\def\sansfont   {}
\def\monofont   {}
\def\CJKmainfont{}
\def\CJKsansfont{}
\def\CJKmonofont{}


%%%%% Style of fonts and line stretch | 字體大小風格及行距 %%%%%
% Base font
\def\baseFontSize{14pt} % valid: 8pt, 9pt, 10pt, 11pt, 12pt, 14pt, 17pt, 20pt
\def\baseLineStretch{1.3} % 行距(倍數)
\def\fakeBoldFactor{2} % 假粗體粗度。(Only affect CJK font.)
% Bibliography
\def\bibFontStyle{\small}
\def\bibLineStretch{1.2}
% Table
\def\floatFontStyle{\small} % content of table
\def\tableLineStretch{1.2}
% Caption
\def\captionFontStyle{\small}
\def\subcaptionFontStyle{\footnotesize}
\def\captionLineStretch{1.2}
% Page
\def\pageHeaderStyle{\sffamily\itshape\footnotesize}
\def\pageFooterStyle{\sffamily\footnotesize}
% listing font style set by basicstyle in \lstdefinestyle, see macro_preamble.tex
% todonotes font style set by textsize option when package loading, see thesis_base.cls . default \footnotesize


%%%%% Really blank page | 純白空白頁 %%%%%
\setboolean{reallyBlankPage}{false} % {true}/{false} true for use  really blank pages between chapters. 


%%%%% Depth | 章節深度 %%%%%
% LaTeX default 2,chapter == 0
\def\secNumDepth{4} % Numbering of sectioning title. 設定章節標題給予數字標號的深度, \paragraph == 4。
\def\tocDepth{2}    % Depth of TOC. 目錄顯示層級,\subsection == 2。


%%%%% Style of titles | 標題風格 %%%%%
\def\titlepageFontFamily{\sffamily}
% set "\rmfamily\CJKfamily{sf}" to mix rm for English and sf(default Kai) for CJK, see wiki on TW_Thesis_Template.
\def\abstractHeaderStyle{\sffamily\Large} % Information on abstract, cls default use \centering

\def\chapterTitleNumStyle   {\sffamily\LARGE\bfseries} % Number of chapter, only for en
\def\chapterTitleStyle      {\sffamily\huge\bfseries} % cls default use \centering for zh
\def\sectionTitleStyle      {\sffamily\Large\bfseries}
\def\subsectionTitleStyle   {\sffamily\large\bfseries}
\def\subsubsectionTitleStyle{\sffamily\normalsize\bfseries}
\def\paragraphTitleStyle    {\sffamily\normalsize\bfseries}
\def\subparagraphTitleStyle {\sffamily\normalsize\bfseries}


%%%%% Style of toc and titles | 目錄及章節風格 %%%%%
% see tutorial. (0 -> original)
\def\tocStyleAlign{0}           % 0/1
\def\tocStyleChapter{1}         % 0/1/2
\def\titleStyle{0}              % 0/1 
    \def\indentBlockSSS{0mm}    % indent \subsubsection{} 
    \def\indentBlockPar{0mm}    % indent \paragraph{}
    \def\indentBlockSPar{0mm}   % indent \subparagraph{}


%%%%% Watermark | 浮水印 %%%%%
% Add a watermark to every page after \startWatermark .
% It will search (image) filename in figures folder.

% Package (\wmMethod) bug / disadvantages:
% background: Background failed in \includepdf page.
% eso-pic: Opacity failed in first image.
\def\wmContent{logo.pdf} % 圖檔名或文字 | image filename or text.
\def\wmMethod{1} % {0}/{1} 0 == background pkg. , 1 == eso-pic pkg.
\def\wmScale{1}                 % 縮放比
\def\wmOpacity{1}               % 透明度
\def\wmShiftFromCenterX{0mm}    % 圖片中心相對紙張中心垂直位移
\def\wmShiftFromCenterY{0mm}    % 圖片中心相對紙張中心水平位移
\def\wmAngle{0}                 % 旋轉角度


%%%%% Misc | 雜項 %%%%%
\setboolean{pdfLinkBoxDisplay}{true} % {true}/{false} ,Draw a box on the link. It only display in pdf viewer, not on paper.
