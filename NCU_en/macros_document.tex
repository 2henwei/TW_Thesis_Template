% 這裡的設定是須要在 \begin{document} 之內才能作用的
\IfStandalone{\standaloneconfig{float=false}}{} % set option for standaloneclass(sub-file)
% for sub-tex, 
% ``float=false'' disable floating environment when build sub-tex.

\fontsize{14}{25}\selectfont  % 可調間距以便閱讀 \fontsize{size}{skip} 15/30 約32每行,兩倍高


% A example for define unit command.
\providecommand{\temp}[1]{$\SI{#1}{\degreeCelsius}$}

\providecommand{\symumm}[1]{$\SI{#1}{\milli\metre}$}
\providecommand{\symums}[1]{$\SI{#1}{\milli\second}$}
\providecommand{\symumM}[1]{$\SI{#1}{\milli\Molar}$}
\providecommand{\symumv}[1]{$\SI{#1}{\milli\volt}$}
\providecommand{\symuma}[1]{$\SI{#1}{\milli\ampere}$}

\providecommand{\abs}[1]{\lvert #1 \rvert } % Required package: amsmath

%%%%%%%% for codes %%%%%%%%%
\definecolor{codegreen}{rgb}{0,0.6,0}
\definecolor{codegray}{rgb}{0.5,0.5,0.5}
\definecolor{codepurple}{rgb}{0.58,0,0.82}
\definecolor{codebgcolor}{rgb}{0.95,0.95,0.92}


\lstdefinestyle{commonStyle}{
    frame=single,
    %backgroundcolor=\color{codebgcolor},
    commentstyle=\color{codegreen},
    keywordstyle=\color{blue},
    numberstyle=\small\color{codegray},
    stringstyle=\color{codepurple},
    basicstyle=\ttfamily\small,
    breakatwhitespace=false,         
    breaklines=true,                 
    captionpos=t,
    caption={\protect\filename@parse{\lstname}\protect\filename@base\text{.}\protect‌​\filename@ext}, 
        % http://tex.stackexchange.com/questions/174541/only-get-filename-and-extension-of-listing-not-whole-path
    keepspaces=true,
    xleftmargin=1cm,
    %xrightmargin=1cm,
    numbers=left,
    numbersep=5pt,
    showspaces=false,                
    showstringspaces=false,
    showtabs=false,                  
    tabsize=4
}

% for console, no line numbers
\lstdefinestyle{consoleStyle}{
    frame=single,
    %backgroundcolor=\color{codebgcolor},   
    commentstyle=\color{codegreen},
    keywordstyle=\color{blue},
    numberstyle=\small\color{codegray},
    stringstyle=\color{codepurple},
    basicstyle=\ttfamily\small,
    breakatwhitespace=false,         
    breaklines=true,                 
    captionpos=t,
    %caption={\protect\filename@parse{\lstname}\protect\filename@base\text{.}\protect‌​\filename@ext}, 
        % http://tex.stackexchange.com/questions/174541/only-get-filename-and-extension-of-listing-not-whole-path
    keepspaces=true,
    xleftmargin=0cm,
    numbers=none,
    showspaces=false,                
    showstringspaces=false,
    showtabs=false,
    tabsize=4
}

\lstdefinestyle{LatexStyle}{
    language={[LaTeX]TeX},
    inputpath={./}, % must same as root tex
    frame=single,
    %backgroundcolor=\color{backcolour},   
    commentstyle=\color{codegreen},
    keywordstyle=\color{blue},
    numberstyle=\small\color{codegray},
    stringstyle=\color{codepurple},
    basicstyle=\ttfamily\small,
    breakatwhitespace=false,         
    breaklines=true,                 
    captionpos=t,
    %caption={\protect\filename@parse{\lstname}\protect\filename@base\text{.}\protect‌​\filename@ext}, 
        % http://tex.stackexchange.com/questions/174541/only-get-filename-and-extension-of-listing-not-whole-path
    keepspaces=true,
    xleftmargin=1cm,
    xrightmargin=1cm,
    numbers=left,
    numbersep=5pt,
    showspaces=false,                
    showstringspaces=false,
    showtabs=false,                  
    tabsize=4
}

\lstset{style=commonStyle} % default style
\renewcommand{\lstlistingname}{Code} % change title of caption to ``Code'' from ``Listing'' 
