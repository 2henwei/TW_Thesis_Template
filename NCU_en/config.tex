%%%%% build setting | 編譯設定 %%%%%
\setboolean{publish}{false} % {true}/{false} Set true before publish. 發怖前設true
% draft option will not load figure, so reduce compile time.
%   add to class option in main.tex ( \documentclass[draft] )
\def\lang{en} % zh/en Set main language.
\setboolean{disableChinese}{true} % {true}/{false} Disable chinese, for English user.
\def\titlepageLang{zh} % zh/en/{} Choose language for titlepage (zh -> Chinese / en -> English / {} -> hide titlepage and disable chinese font)
\synctex=1 % Enable SyncTeX

%%%%% Information of your document. | 定義文件資訊 %%%%%
\def\deptshort {物理}
\def\dept      {物理研究所}%XX學系XXX碩士班(請參考「中央大學學位論文撰寫體例參考」附錄)
\def\degree    {碩士} % Master(碩士)/Ph.D(博士)
\def\titleZh   {中央大學論文樣板(英文)}   % Chinese title
\def\titleEn   {NCU Thesis Template(English)} % English title
\def\title     {\titleZh} % Main title, default Chinese title
\def\subtitle  {\titleEn} % Subtitle, default English title, empty allowed.
\def\logo      {}   % 填入校徽檔名。中央無校徽在封面,維持空白可除去校徽
\def\author    {Your Name}
\def\mprof     {Your committee }
\def\sprofi    {Committee member} % 共同指導 1
\def\sprofii   {}      % 共同指導 2                  
\def\degreedateZh{中~華~民~國~一百零五~年~六~月} % Only for Chinese titlepage
\def\degreedateEn{June 2016}      % Only for English titlepage
\def\copyyear  {2012-2013}

\setboolean{printcopyright}{false} % {true}/{false} print copyright text on titlepage or cover.

\def\keywordsZh{關鍵字, 論文, 樣板, 讓我畢業} % don't use \def\keywords
\def\keywordsEn{Keyword, Thesis, Template, Graduate me}



%%%%%% set OS | 設定作業系統 %%%%%
% Will overwrite \OS if your latex compile parameter has ``-shell-escape''(Texstudio default enable).
\def\OS{linux} % {linux}/{win}, only effect auto select CJK font.(CJK means Chinese, Japanese, and Korean)

%%%%%% set font | 設定中英文字型 %%%%%
% keep empty for default font. CJK font must set OS for auto select.
% Setting by name of font(English will be better) or font filename.
% Linux 利用指令 fc-list :lang=zh 來查詢可以用的字體名稱。
% Windows 使用內建字型管理查詢。
\def\mainfont   {} % default use Latin Modern Roman (lmodern pkg.)
\def\sansfont   {}
\def\monofont   {}
\def\CJKmainfont{}
\def\CJKsansfont{}
\def\CJKmonofont{}

%%%%% depth | 文獻列表風格 %%%%%
\setboolean{bibStyleNameYear}{false} % {true}/{false} true for use name,year to sort and cite.(If you want use custom option in .cls, set false here. )

%%%%% depth | 章節深度 %%%%%
% LaTeX 預設2,chapter == 0。
\def\secNumDepth{4} % 設定章節標題給予數字標號的深度, \paragraph == 4。
\def\tocDepth{2}    % 目錄顯示層級,\subsection == 2。

%%%%% style of toc and titles | 目錄及章節風格 %%%%%
% see tutorial. (0 -> original)
\def\tocStyleAlign{0}           % 0/1
\def\tocStyleChapter{1}         % 0/1/2
\def\titleStyle{0}              % 0/1 
    \def\indentBlockSSS{0mm}    % indent \subsubsection{} 
    \def\indentBlockPar{0mm}    % indent \paragraph{}
    \def\indentBlockSPar{0mm}   % indent \subparagraph{}

%%%%% style of fonts and line stretch | 字體大小風格及行距 %%%%%
% Base font
\def\baseFontSize{12pt} % allowed: 8pt, 9pt, 10pt, 11pt, 12pt, 14pt, 17pt, 20pt
\def\baseLineStretch{1.3}
% Bibliography
\def\bibFontStyle{\small}
\def\bibLineStretch{1.2}
% Table
\def\floatFontStyle{\small} % content of table
\def\tableLineStretch{1.2}
% Caption
\def\captionFontStyle{\small}
\def\subcaptionFontStyle{\footnotesize}
\def\captionLineStretch{1.2}
% Page
\def\pageHeaderStyle{\sffamily\itshape\footnotesize}
\def\pageFooterStyle{\sffamily\footnotesize}
% listing font style set by basicstyle in \lstdefinestyle, see macro_preamble.tex
% todonotes font style set by textsize option when package loading, see cls. default footnotesize

%%%%% Style of titles | 標題風格 %%%%%
\def\abstractHeaderStyle{\sffamily\Large} % Information on abstract, cls default use \centering

\def\chapterTitleNumStyle{\sffamily\LARGE\bfseries} % Number of chapter, only for en
\def\chapterTitleStyle{\sffamily\huge\bfseries} % cls default use \centering for zh
\def\sectionTitleStyle{\sffamily\Large\bfseries}
\def\subsectionTitleStyle{\sffamily\large\bfseries}
\def\subsubsectionTitleStyle{\sffamily\normalsize\bfseries}
\def\paragraphTitleStyle{\sffamily\normalsize\bfseries}
\def\subparagraphTitleStyle{\sffamily\normalsize\bfseries}

%%%%% Misc | 雜項 %%%%%
\setboolean{pdfLinkBoxDisplay}{true} % {true}/{false} ,Draw a box on the link. It only display in pdf viewer, not on paper.
\def\fakeBoldLevel{2} % 假粗體粗度。(Only affect CJK font.)