\documentclass[class=NCU_thesis, crop=false]{standalone}
\begin{document}
\chapter{LibreOffice Draw}
\label{sec:a_loDraw}
因為朋友的M\$ Office 無法匯出eps
\footnote{M\$ 好像很不重視eps(還是說像NTFS,不想給死老百姓用?),預設軟體都無法讀取。}
,於是附上我製作eps圖的方法。當然你如果有其它更好用、更順手的軟體,你就直接用那個吧!
首先你要有LibreOffice,沒有的請去下載。\url{https://zh-tw.libreoffice.org/}

LibreOffice 如同M\$ Office有各種不同的子軟體
\footnote{不過和M\$O不同的是 LO 各軟體共用同一核心,所以很多功能是通用的。}
,其中Draw 就是專門繪圖、製造海報的工具。
當你畫完圖後,點滑鼠左鍵選擇欲匯出物件(按住Shift連選,或是Ctrl+A全選)並於工具列選擇「匯出」(\cref{fig:loDraw_select})。
存檔視窗中,副檔名使用「.eps」或是選擇類型處點選「eps」來指定匯出為eps格式。記得勾選左下角「選取」框限制僅匯出選取的物件,否則會將整張紙面一起匯出。(\cref{fig:loDraw_save})
按下確定存檔後會出現eps格式的設定細節。基本上不須要改,頂多注意要使用彩色還是灰階與是否壓縮。(\cref{fig:loDraw_epsOpt})
\insfig[0.7][fig:loDraw_select]{LibreOffice_Draw_select.png}
\insfig[0.8][fig:loDraw_save]{LibreOffice_Draw_epsSave.png}
\insfig[0.5][fig:loDraw_epsOpt]{LibreOffice_Draw_epsOption.png}

\end{document}
