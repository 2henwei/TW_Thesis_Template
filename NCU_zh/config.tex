%%%%% build setting | 編譯設定 %%%%%
\setboolean{publish}{false} % {true}/{false} Set true before publish. 發怖前設true
% draft option add to \documentclass[draft] in main.tex

\synctex=1 % 啟用SyncTeX

%%%%% Information of your document. | 定義文件資訊 %%%%%
\def\deptshort {物理}
\def\dept      {物理研究所}%XX學系XXX碩士班(請參考「中央大學學位論文撰寫體例參考」附錄)
\def\degree    {碩士} % 碩士/博士
\def\titleZh   {中央大學LaTeX論文樣板(中文)}   % Chinese title
\def\titleEn   {NCU LaTeX Thesis Template(Chinese)} % English title
\def\title     {\titleZh} % Main title, default Chinese title
\def\subtitle  {\titleEn} % Subtitle, default English title, empty allowed.
\def\logo      {}   % 填入校徽檔名。中央無校徽在封面,維持空白可除去校徽
\def\author    {君の名は}
\def\mprof     {你的指導教授}
\def\sprofi    {你的共同指導} % 共同指導 1
\def\sprofii   {}      % 共同指導 2                  
\def\degreedate{中~華~民~國~一百零五~年~六~月}
\def\copyyear  {2012-2013}

\setboolean{printcopyright}{false} % {true}/{false} print copyright text on titlepage or cover.

\def\keywordsZh{關鍵字, 論文, 樣板, 讓我畢業} % don't use \def\keywords
\def\keywordsEn{Keyword, Thesis, Template, Graduate me}



%%%%%% set OS | 設定作業系統 %%%%%
% Will overwrite \OS if your latex compile parameter has ``-shell-escape''(Texstudio default enable).
\def\OS{linux} % {linux}/{win}, only effect auto select CJK font.(CJK means Chinese, Japanese, and Korean)

%%%%%% set font | 設定中英文字型 %%%%%
% keep empty for default font. CJK font must set OS for auto select.
% Linux 利用指令 fc-list :lang=zh 來查詢可以用的字體名稱。
\def\mainfont   {}
\def\sansfont   {}
\def\monofont   {} %DejaVu Sans Mono
\def\CJKmainfont{}
\def\CJKsansfont{}
\def\CJKmonofont{}

%%%%% depth | 章節深度 %%%%%
% LaTeX 預設2,chapter == 0。
\setcounter{secnumdepth}{4} % 設定章節標題給予數字標號的深度, \paragraph == 4。
\setcounter{tocdepth}{2}  % 目錄顯示層級,\subsection == 2。

%%%%% style of toc and titles | 目錄及章節風格 %%%%%
% see tutorial.
\def\tocStyle{0}
\def\titleStyle{0}
    \def\indentblocksss{0mm}    % indent \subsubsection{}
    \def\indentblockpar{0mm}    % indent \paragraph{}